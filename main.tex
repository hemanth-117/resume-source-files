\documentclass[12 pt]{article}
\usepackage[a4paper,bottom = 10mm,left = 0.56in,right = 0.56in,top = 6.4mm]{geometry}
\usepackage{graphicx}
\usepackage{amsmath}
\usepackage{array}
\usepackage{enumitem}
\usepackage{wrapfig}
\usepackage{microtype}
\usepackage{titlesec}
\usepackage{textcomp}
\usepackage[hidelinks]{hyperref}
\usepackage{xcolor}
\usepackage{verbatim}
\newcommand{\xfilll}[2][1ex]{
\dimen0=#2\advance\dimen0 by #1
\leaders\hrule height \dimen0 depth -#1\hfill}
\titleformat{\section}{\large\scshape\raggedright}{}{0em}{}
\renewcommand\labelitemi{\raisebox{0.4ex}{\tiny$\bullet$}}
\renewcommand{\labelitemii}{$\cdot$}
\pagenumbering{gobble}
\hypersetup{
    colorlinks=true,
    linkcolor=blue,
    filecolor=magenta,      
    urlcolor=blue,
    % pdftitle={Overleaf Example},
    % pdfpagemode=FullScreen,
    }
\begin{document}
\vspace*{50mm}
\hspace{-14pt}Pursuing {\bf Minor} in \textbf{Machine Intelligence and Data Science}
\vspace{-10pt}
\section*{\textcolor{blue}{\LARGE Scholastic Achievements}\xfilll[0pt]{0.5pt}}
\vspace{-8pt}
\begin{itemize}[itemsep = -0.75 mm, leftmargin=*]
\item Secured \textbf{All India Rank 232} in \textbf{JEE Advanced} among over 150 thousand candidates \hfill{\sl \small(2021)}
		\item Achieved \textbf{All India Rank 436} in \textbf{JEE Main} among over 1.2 million candidates \hfill{\sl \small(2021)}
		\item Secured \textbf{State Rank 48} in AP-EAMCET conducted by Andhra Pradesh State Council \hfill{\sl \small (2021)}
        \item Secured \textbf{State Rank 185} in TS-EAMCET conducted by Telangana State Council \hfill{\sl \small (2021)}
		\item Secured \textbf{All India Rank 364} in \textbf{KVPY} among over 50 thousand candidates \hfill{\sl \small(2020)}
\end{itemize}
%\vspace{-19pt}

\vspace{-22pt}
\section*{\textcolor{blue}{\LARGE Key Projects}\xfilll[0pt]{0.5pt}}
{\fontfamily{lmss}\selectfont
		\textbf{{Cinema A to Z}}\hfill{\sl \small (Autumn 2022)}\\
	}{\it Guide: Prof. Kavi Arya \textbar} {\it Ongoing Course Project}\hfill{\sl \small IIT Bombay}\\
	\begin{itemize}[itemsep = -0.8 mm, leftmargin=*]
		\vspace{-16pt}
		\item Target to build a one stop user-friendly solution \textbf{Web application} which will help the user by giving all the ratings, cast, category/genre, user reviews
		\item Using \textbf{Web-scraping} to extract reviews from IMDb, Rotten Tomatoes, and store in \textbf{SQL database}
		\item Implementing a \textbf{Networking} system to share data resource with multiple applications and users
		\item  Target to use \textbf{HTML, CSS, Bootstrap, Javascript} to make Web application more responsive
	\end{itemize}
{\fontfamily{lmss}\selectfont
		\textbf{{Rail Planner}}\hfill{\sl \small (Autumn 2022)}\\
	}{\it Guide: Prof. Supratik Chakraborty \textbar} {\it Ongoing Course Project}\hfill{\sl \small IIT Bombay}\\
	\begin{itemize}[itemsep = -0.8 mm, leftmargin=*]
		\vspace{-16pt}
		\item Developed software which integrates booking a journey, viewing and managing current journeys, and maintaining reviews for all journeys, taking inspiration from apps like \textbf{Eurail} and \textbf{IRCTC}
		\item Implemented multiple data structures in C++, including \textbf{Linked Lists, Dictionaries, Binary Search Trees, Tries}, and \textbf{Priority Queues}, and utilized each of them within suitable components
		\item implemented the \textbf{Knuth-Morris-Pratt algorithm} for pattern matching, searching in strings.
		\item Created an \textbf{admin interface} to add stations and journeys, and display the current database
		\item Created a \textbf{user interface} to query journeys, look for reviews using \textbf{ Maximum Priority Queue} to filter the reviews above a \textbf{threshhold value}, plan journeys, and write reviews for past journeys
		%\item For a \textbf{Doublylinkedlist} with n elements, using additional \textbf{storage space} of amount \textbf{n/K}, for a given \textbf{K}, choosing pivot in atmost \textbf{O(K)} time to impliment \textbf{QuickSort} algorithm.
	\end{itemize}
	{\fontfamily{lmss}\selectfont
		\textbf{{Tic-Tac-Toe}}\hfill{\sl \small (Autumn 2022)}\\
	}{\it Guide: Prof. Kavi Arya \textbar} {\it Course Project: Software Systems Lab}\hfill{\sl \small IIT Bombay}\\
	\begin{itemize}[itemsep = -0.8 mm, leftmargin=*]
		\vspace{-16pt}
		\item Developed a two-player tic-tac-toe using \textbf{Socket Programming, inter-process communication}
		\item The game is played between two players with players entering their commands in different terminals
		\item Future plan to enable \textbf{remote play} between two players with one of the players starting the server
	\end{itemize}
	%\newpage
	{\fontfamily{lmss}\selectfont
		\textbf{{Portfolio Website}}\hfill{\sl \small (Autumn 2022)}\\
	}{\it Guide: Prof. Kavi Arya \textbar} {\it Course Project: Software Systems Lab}\hfill{\sl \small IIT Bombay}\\
	\begin{itemize}[itemsep = -0.8 mm, leftmargin=*]
		\vspace{-16pt}
		\item Developed  a \textbf{responsive} website containing various pages describing me and my details
		\item Used \textbf{HTML, CSS, Javascript and Bootsrap} in the website to make it responsive and interactive
		\item Hosted the website on IITB CSE servers by adding source files to the server using \textbf{SSH}
	\end{itemize}
\vspace{-16pt}
\section*{\textcolor{blue}{\LARGE Other Projects}\xfilll[0pt]{0.5pt}}
	{\fontfamily{lmss}\selectfont
		\textbf{{Random Walkers}}\hfill{\sl \small (Autumn 2022)}\\
	}{\it Guide: Prof. Suyash P. Awate \textbar} {\it Course Project: Data Analysis and Interpretation}\hfill{\sl \small IIT Bombay}\\
	\begin{itemize}[itemsep = -0.8 mm, leftmargin=*]
		\vspace{-16pt}
		\item Simulated N random walkers in \textbf{Python}, and obtained the \textbf{Gaussian Distribution plot} of their final locations using \textbf{Matplotlib} 
 and verified the \textbf{Law of Large Numbers} by analysing the true and empirically computed mean and variance
	\end{itemize}
	{\fontfamily{lmss}\selectfont
		\textbf{{Maze Ball}}\hfill{\sl \small (Spring 2022)}\\
	}{\it Guide: Prof. Rushikesh K. Joshi \textbar} {\it Course Project : Abstractions and Paradigms}\hfill{\sl \small IIT Bombay}\\
	\begin{itemize}[itemsep = -0.8 mm, leftmargin=*]
		\vspace{-16pt}
		\item Programmed an Maze Ball with the help of \textbf{FLTK library} using \textbf{Inheritance} and \textbf{Event Handling}
	\end{itemize}
	{\fontfamily{lmss}\selectfont
		\textbf{{Image Generation using PCA}}}\hfill{\sl \small (Autumn 2022)}\\
{\it Guide: Prof. Suyash P. Awate $|$ Course Project: Data Analysis and Interpretation} \hfill{\it IIT Bombay}
\\\vspace{-15pt}
\begin{itemize}[itemsep = -0.5 mm, leftmargin=*]
	\item \textls[15]{Performed \textbf{dimensionality reduction, hyperplalne} fitting on fruit images and MNIST dataset using \textbf{PCA} in Python}, \textls[21]{Implemented a \textbf{generative model}, hence sampling some unseen data from the above implementation of hyperplane}
\end{itemize}
\vspace{\baselineskip}
\vspace{-13pt}

\section*{\textcolor{blue}{\LARGE Technical Skills}\xfilll[0pt]{0.5pt}}
\vspace{-8pt}
\begin{tabular}{p{4.5cm} p{13.5cm}}
  \textbf{Programming:} &  C++, C, Python, Java, Bash, Awk, Sed \\
  \textbf{Web Development:} & HTML, CSS, Bootstrap, JavaScript\\
  \textbf{Software:} &  Git, \LaTeX\\
  \textbf{Packages:} & NumPy, Matplotlib, FLTK\\
  \textbf{Documentation:} & Doxygen, Sphinx\\
\end{tabular}
 \section*{\textcolor{blue}{\LARGE Courses Undertaken}\xfilll[0pt]{0.5pt}}
\vspace{-12pt}
\hspace{-5pt}
  \begin{tabular}{p{40mm} p{13.2cm}}
    \vspace{-3pt}\textbf{Computer Science:} & \vspace{-5pt}Data Structures and Algorithms + Lab*, Discrete Structures*, Data Analysis and Interpretation*, Software Systems Lab*, Design and Analysis of Algorithms**, Digital Logic Design + Lab**, Computer Networks + Lab**, Logic for Computer Science**, Abstractions and Paradigms in Programming, Computer Programming and Utilization\\
    \vspace{-5pt}\textbf{Mathematics:} & \vspace{-5pt}Calculus, Linear Algebra, Differential Equations\\
    \vspace{-5pt}\textbf{Others:} & \vspace{-5pt}Introduction to Electrical and Electronics Circuits*, Quantum Physics and Application, Basics of Electricity and Magnetism, Engineering Graphics and Drawing, Physical Chemistry, Organic and Inorganic Chemistry, Biology\\
  \end{tabular}
  \vspace{1pt}\\
  \hspace*{120mm}{\sl *{\it to be completed in Autumn 2022}}\\
  \hspace*{120mm}{\sl **{\it to be completed in Spring 2023}}
\vspace{-10pt}

\section*{\textcolor{blue}{\LARGE Extracurricular}\xfilll[0pt]{0.5pt}}
\vspace{-8pt}

% {\bf \large Creative Field}  \vspace{-4pt}
\begin{itemize}[itemsep = -0.75 mm, leftmargin=*]
\vspace{-2pt}
	\item Successfully completed a course under the \textbf{National Sports
	Organization(NSO)}
	\hfill{\sl (Autumn 2021)}
	\end{itemize}
	{\fontfamily{lmss}\selectfont
		\textbf{{\LaTeX{} Bootcamp}}
	}\hfill{\sl (Summer 2022)}\\[-1pt]
	{\sl \textit{Learner’s Space \textbar }{\textit{ IIT Bombay}}}\\
	\begin{itemize}[itemsep = -0.8 mm, leftmargin=*]
		\vspace{-16pt}
		\item Implemented \LaTeX{} software to typeset technical and mathematical documents using a \textbf{workflow}
		\item Successfully completed \textbf{3} assignments based on \textbf{text formatting}, designing a \textbf{résumé} and typesetting
of a \textbf{technical document} having equations, matrices, tables and lines of programming languages
	\end{itemize}
	{\fontfamily{lmss}\selectfont
		\textbf{{RC Plane}}}
	 \hfill{\sl \small (Spring 2022)}\\
{\it Aeromodelling Club}\hfill{\sl \small IIT Bombay}\\\vspace{-16pt}
\begin{itemize}[itemsep = -0.75 mm, leftmargin=*]
\item Constructed a remote controlled plane out of depron sheets keeping in mind the wing loading, wing shape, air drag, balance and weight for a perfect take off, flight and landing
  \end{itemize}
  \vspace{-2pt}
\vspace{-2pt}
\end{document}
